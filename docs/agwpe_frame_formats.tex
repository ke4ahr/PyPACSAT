\documentclass[11pt,a4paper]{article}
\usepackage[utf8]{inputenc}
\usepackage[T1]{fontenc}
\usepackage{lmodern}
\usepackage{geometry}
\geometry{margin=1in}
\usepackage{booktabs}
\usepackage{array}
\usepackage{caption}
\usepackage{longtable}
\usepackage{fancyhdr}
\usepackage{lastpage}
\usepackage{xcolor}
\usepackage{listings}
\usepackage{inconsolata}

\lstset{
  basicstyle=\ttfamily\small,
  breaklines=true,
  frame=single,
  numbers=left,
  numberstyle=\tiny,
}

\title{AGWPE Frame Formats (Updated for Current Implementation)}
\author{Kris Kirby, KE4AHR}
\date{December 27, 2025}

\pagestyle{fancy}
\fancyhf{}
\rhead{AGWPE Frame Formats}
\lhead{PACSAT Revival Project}
\cfoot{Page \thepage\ of \pageref{LastPage}}

\begin{document}

\maketitle

\section{Overview}

The AGW Packet Engine (AGWPE) TCP/IP API uses a fixed \textbf{36-byte header} followed by a variable-length data payload. All frames share the same header structure, with the meaning determined by the \texttt{DataKind} byte.

The current Python implementation supports unproto UI frames (primary for PACSAT), connected mode, raw unproto, and outstanding frame queries.

\section{Universal Frame Header (36 bytes)}

\begin{longtable}{>{\raggedleft\arraybackslash}p{2cm} p{1.5cm} p{4cm} p{8cm}}
\toprule
\textbf{Offset} & \textbf{Size} & \textbf{Field} & \textbf{Description} \\
\midrule
\endhead
0 & 1 byte & Port & Radio port number (0--255). Port 0 is special (general commands). \\
1--3 & 3 bytes & Reserved & Always 0x00. \\
4 & 4 bytes & DataKind & Frame type (single ASCII character, e.g., `R', `G', `X', `M', `D', `K', `Y', `C', `c', `d', `D'). Little-endian. \\
8 & 10 bytes & CallFrom & Source callsign, left-justified, space-padded (max 9 characters + space padding). \\
18 & 10 bytes & CallTo & Destination callsign, same format as CallFrom. \\
28 & 4 bytes & DataLen & Length of payload data (little-endian 32-bit integer). \\
32 & 4 bytes & Reserved & Always 0x00. \\
\bottomrule
\caption{AGWPE Universal Frame Header}
\end{longtable}

\textbf{Payload}: Immediately follows the header, length = DataLen bytes.

\section{Implemented Frame Types}

\begin{longtable}{l l l p{8cm}}
\toprule
\textbf{DataKind} & \textbf{Name} & \textbf{Direction} & \textbf{Status / Notes} \\
\midrule
\endhead
`R' & Register & Client $\to$ Server & Implemented – Required at startup \\
`G' & Version & Both & Implemented – Version query/response \\
`X' & Ports & Both & Implemented – Port capabilities \\
`M' & Monitor & Client $\to$ Server & Implemented – Enable monitoring \\
`D' & Unproto Data & Both & Fully implemented – Primary for PACSAT (PID in payload) \\
`K' & Raw Unproto & Client $\to$ Server & Implemented – No PID byte \\
`Y' & Outstanding Frames & Both & Implemented – Query pending frames \\
`C' & Connect (SABM) & Client $\to$ Server & Implemented – Connected mode support \\
`c' & Confirm (UA) & Server $\to$ Client & Implemented \\
`d' & Connected Data & Both & Implemented – I-frames \\
`D' & Disconnect (DISC) & Both & Implemented \\
\bottomrule
\caption{Implemented AGWPE Frame Types}
\end{longtable}

\section{Key Implementation Notes}

- **Unproto ('D')**: Primary mode for PACSAT – payload = PID byte + Info field
- **Connected Mode**: Fully supported but disabled by default (PACSAT uses unproto only)
- **Raw ('K')**: Allows sending frames without PID byte
- **Multiple Applications**: Not supported (single application only – FIXME present in code)
- **Monitoring**: 'M' frame enables reception of all traffic on a port

\section{References}

- AGWPE TCP/IP API Tutorial by Pedro E. Colla (LU7DID) and George Rossopoulos (SV2AGW), 2000
- Current implementation in PyAGW3/agwpe.py (PACSAT Revival Project, 2025)

\end{document}
